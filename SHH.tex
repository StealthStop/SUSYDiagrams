
%%%%%%%%%%%%%%%%%%%%%% Feynman diagram for SHH

\documentclass{article}

\usepackage{feynmp-auto}

%%%%%% Custom commands
\def\chigen{$\mathrm{\widetilde{\chi}}$}
\def\chiz{$\mathrm{\widetilde{\chi}^0_1}$}
\def\chitz{$\mathrm{\widetilde{\chi}^0_{2}}$}
\def\chipm{$\mathrm{\widetilde{\chi}^\pm_{1}}$}
\def\chimp{$\mathrm{\widetilde{\chi}^\mp_{1}}$}
\def\sGlu{$\mathrm{\widetilde{g}}$}
\def\sQ{$\mathrm{\widetilde{q}}$}
\def\sQb{$\mathrm{\overline{\widetilde{q}}}$}
\def\sBot{$\mathrm{\widetilde{b}_1}$}
\def\sBotb{$\mathrm{\overline{\widetilde{b}}_1}$}
\def\sTop{$\mathrm{\widetilde{t}_1}$}
\def\stop{$\mathrm{\widetilde{t}}$}
\def\sTopb{$\mathrm{\overline{\widetilde{t}}_1}$}
\def\stopb{$\mathrm{\overline{\widetilde{t}}}$}
\def\sTopt{$\mathrm{\widetilde{t}_2}$}
\def\sToptb{$\mathrm{\overline{\widetilde{t}}_2}$}
\def\sGra{$\mathrm{\widetilde{G}}$}
\def\sS{$\mathrm{\widetilde{S}}$}
\def\PH{$\mathrm{H}$}
\def\PZ{$\mathrm{Z}$}
\def\PW{$\mathrm{W^\pm}$}
\def\PV{$\mathrm{V}$}
\def\Ph{$\mathrm{h}$}
\def\Pl{$\mathrm{\ell}$}
\def\Pvl{$\mathrm{\nu_\ell}$}
\def\Pavl{$\mathrm{\overline{\nu}_\ell}$}
\def\sLep{$\widetilde{\ell}$}
\def\sNu{$\widetilde{\nu}$}
\def\sTau{$\widetilde{\tau}$}
\newcommand{\anti}[1]{$\mathrm{\overline{#1}}$}
\newcommand{\plus}[1]{$\mathrm{#1^+}$}
\newcommand{\minus}[1]{$\mathrm{#1^-}$}
\newcommand{\susy}[1]{$\mathrm{\widetilde{#1}}$}


\def\MainQuark{t}


%%%%%%%%%%%%%%%%%%%%%%%%%%% Document %%%%%%%%%%%%%%%%%%%%%%%%%%%
\begin{document}
\thispagestyle{empty}


%%%%%%%% THE NAME OF THE fmffile HAS TO BE ``Feynman<filename>'' TO USE compile.py %%%%%%%%%%%%%%%
\begin{fmffile}{FeynmanSHH}
\parbox{500mm}{
    
\begin{fmfgraph*}(180,140) %\fmfpen{thick}
  \fmfset{arrow_len}{cm}\fmfset{arrow_ang}{0}
  
  %%%%%%%%%%%% Specifying number of inputs/outputs
  \fmfleftn{i}{2}
  \fmfrightn{o}{8}
  \fmflabel{}{i1}
  \fmflabel{}{i2}
    
  %%%%%%%%%%%% Incoming protons (one line)
  \fmf{fermion, tension=22., lab=p, label.side=right}{v1,i1}
  \fmf{fermion, tension=22., lab=p, label.side=left}{v1,i2}
           
  %%%%%%%%%%%% Produced SUSY particles
  \fmf{dashes, tension=15., label=\stopb, label.side=right}{v1,v2}
  \fmf{dashes, tension=15., label=\stop, label.side=left}{v1,v3}
  
  %%%%%%%%%%%%% Decays and vertex circles

  %% 1st stop decay
  \fmf{vanilla, tension=7., label=\susy{S}, label.side=right}{v2,v4}
  \fmf{boson, tension=3.}{v2,v4}
  %\fmf{gluon, tension=3.}{v2,o4}
  %\fmflabel{g}{o4}
  
  %% 1st singlino decay
  \fmf{dashes, tension=3., label=S, label.side=left}{v4,v6}

  \fmf{fermion, tension=7.}{v4,o1}
  \fmflabel{\susy{G}}{o1}
  
  \fmf{fermion}{v6,o2} 
  \fmflabel{\anti{b}}{o2}
  
  \fmf{fermion}{v6,o3} 
  \fmflabel{b}{o3}
  
  \fmf{fermion}{v2,o4}
  \fmflabel{$\mathrm{\overline{\MainQuark}}$}{o4}
  
  %% 2nd stop decay
  \fmf{vanilla, tension=7., label=\susy{S}, label.side=left}{v3,v5}
  \fmf{boson, tension=3.}{v3,v5}
  %\fmf{gluon, tension=3.}{v3,o6}
  %\fmflabel{g}{o6}

  %% 2nd singlino decay
  \fmf{dashes, tension=3., label=S, label.side=right}{v5,v7}

  \fmf{fermion}{v3,o5}
  \fmflabel{\MainQuark}{o5}

  \fmf{fermion}{v7,o6} 
  \fmflabel{\anti{b}}{o6}

  \fmf{fermion}{v7,o7}
  \fmflabel{b}{o7}

  \fmf{fermion, tension=7.}{v5,o8}
  \fmflabel{\susy{G}}{o8}
  
  %% Vertex circles
  \fmfdot{v2,v3,v4,v5,v6,v7}
  
  %%%%%%%%%%%% Additional lines on incoming protons and blob
  \input{shared/protons.tex}
  
\end{fmfgraph*}

}           
\end{fmffile} 

\end{document}
